\chapter{Conclusion and Future Work}
\section{Conclusion}
An adaptive AI that doesn't cheat can beat a 'static' AI that does. However care must be taken when
adjusting the priorities, and it can be rather hard to adjust the priorities to work for 'all'
situations. With a better placement of buildings and attack algorithms it would probably be
overwhelming for a human player to keep up with the micromanagement which would certainly be
implemented in a real product.

\subsection{Project/Archiceture conclusion}
Creating a good architecture\index{architecture} that is maintainable and
flexible from the beginning and plan for extra functionality helped us
enourmously. This made it very easy for us to extend the functionality of
many components when we needed. If a new feature was added a change was often
only require in one or two classes, seldom more than two. See
\ref{sec:architecture} for a class diagram of the architecture.

We have both gained enormous experience working with this project; we understand a large amount of
how the Spring API works, how much work and how hard it is to generate priorities for
units---especially attacking units, how a task system can be implemented and used which isn't
specific for the RTS genre, and a lot of other AI and non-AI related topics.

\subsection{Things we would've done differently}
In fact there aren't much that we wouldn't do the same. However there is one area that Al Ice isn't
very good of, performance. It uses a lot more of CPU than other bots, that is mainly because most
parts of the system isn't really optimized at all. However that wasn't a part of the
problem, and with more time some of the performance problems when running with a faster
speed could have been fixed. Playing as a human against Al Ice in normal speed wasn't any problem though.


\section{Future work}
There are many parts of Al Ice that can either be improved, added, or replaced entirely by something
else. You can probably find several features that can be a base for thesis work.

\subsection{Improvements to the current system}
\subsubsection{Priority System} 
\label{sec:priority_further_improvements}
\paragraph{Optimized priority generation}
Instead of generating priorities for all the units when we have an available
builder it would be better to only generate priorities for the units that the
available builder can build.

\paragraph{More accurate unit priority}
\label{sec:priority_improvement_accurate_unit_priority}
The current priority system does not take into account our own DPS into the
calculation of the damage type in the attack force priority, and armored
buildings priority generation. The effect of this would be that we create many
units that are good against the enemy's armor type, and when we have enough it
would start to create units that are good against the second most used armor
type. To get a more variety the units priority currently decreases with the
number of units that we already have. However this will not make other units
which share the same damage type get lower priority which is the desired effect.

For anti-air units there could be some sort of threshold. I.e. Al Ice should always try to have a
certain amount of anti-air units, even when no air units have been spotted. This base amount of
anti-air units should get higher priority if an air unit has been seen. When the threshold is passed
the calculation is done as now; anti-air units get a bonus priority depending on the size of the
enemies' air units.

Another approach is to insert a new armor type, flying, and a new damage type, anti-air. The damage
multipliers for the all other damage types will be skipped in the priority calculation. The
effect of this would make anti-air units have the lowest armor type health if they only consider the
air units making them highest priority to create. However this functionality needs to be implemented
with the improvement discussed above where our own DPS is taken into account. If it isn't taken into
account it will make Al Ice to literally spit out anti-air units.

\paragraph{Better economics priority over time}
\label{sec:priority_improvement_better_economics_priority}
Currently Al Ice will always try to increase it's current income linearly for metal and almost
linearly for energy. However this algorithm has one major flaw, when an enemy destroys lots of
resource buildings it will most certain make the AI to create resource buildings it top
priority rather than creating a balanced amount of defensive and recourse buildings.

\subsubsection{Task system and tasks}
\index{task}\index{task system}
\paragraph{More than one high-level task per task unit}
Instead of only having one high-level task task units could have the ability to have different
high-level tasks with priorities and a set of 'normal' tasks for each high-level task. This would
have the similar effect of stacking tasks. E.g. a task-unit could have a high-level task 'attack'
with a normal priority; the active tasks being MoveCloseTo (normal priority), and AttackTarget
(high priority). Maybe the enemy decides to attack and our unit is close to the attacking position;
making the General assign a new high-level task 'defend' with high priority to the unit; this would
push the current stack of 'normal' tasks and a new empty set would be available where the unit would
get a new MoveCloseTo task and AttackTarget task.

\paragraph{Base attack}
\index{base attacks}
\subparagraph{'Checkpoints' to the target destination in base attack}
The base attack task could be improved by having several regroup positions
on the way to the target destination---now the units are scattered in a fine line due to the fact that units have different
moving speed.

\subparagraph{Advanced grouping}
Instead of sharing the same attack target for all attacking units it could be split into the type of
the damage type the unit has; creating one list for each kind. Then changing the
\texttt{getcloseEnemy()} to \texttt{getCloseEnemy(damageType)}; this would make the derived classes
able to chose different targets, i.e. those that are effective, for the different kinds of damage
types---making Al Ice destroy targets quicker.

The intention was that the attack target task should look for the best
suitable enemy; e.g. if there are several enemy units in attack range, look for an enemy with an armor type that is weak to the
units damage type. This way Al Ice would destroy it's enemies much faster and
with minimal loss.
 
\subsection{New Features}
\subsubsection{Better path finding} Al Ice path finding and movement system relies on the path finding
implement in Spring which is not advance at all; it only has a path finding for the terrain, but
not the buildings placed on the map. Units using the internal Spring path finding can easily get
stuck, if they try to move in the same direction they will still be stuck and might make more units
get stuck. By implementing an overlay on the base path finding to include building avoidance, i.e.
not try to move straight through buildings, many functions would function better.

\subsubsection{Advanced terrain analysis}
An implementation of advanced terrain analysis would probably improve the probability to beat the
enemy. Flanking positions is one good example that would be useful, but most of all is probably
where to build defensive (armored) buildings and regular buildings. A test could be conducted to
battle this version of Al Ice and how important terrain analysis is.

\subsubsection{Dynamic unit priority}
One could implement a dynamic unit priority. 
Dynamic meaning a unit priority system that would work on every modification
based on the Spring engine. This will probably make the AI more attractive because of the popularity
of the other mods.

\subsubsection{Task priority and planning}
Right now Al Ice does not plan or
care about the future of the game, all priority calculation is just for the
current moment and the General does only try to use all the resources at all
time. If one implemented a task priority and a planner for the General. The planner could also be
implemented without a pre-generated task priority, it would just analyze the outcome of executing
the task and then adding a priority to the task.

\subsubsection{Lua support}
With Lua support it would be easy for more advanced 'users' to implement more functionality into Al
Ice, mostly through tasks. With a planning system it would be relatively easy create new tasks and
add them in the set used by Al Ice. It would also make the user able to customize its behaviour by
maybe removing or replacing tasks.

\subsubsection{Various team play features}
As of now Al Ice will act as a stubborn child if a player or an AI plays with it; it wants
everything for itself. A new feature could be the ability to play with others in the same team.
Three different approaches could be used; One for interacting with other team-members that also are
Al Ice, i.e. the two AI-players would act more or less as one; another approach when playing with
other AIs probably that the other AI is the 'master', i.e. when it attacks Al Ice attacks; and the
last approach when interacting with humans. The use of chat would be a great idea of interaction,
whereas the player could assign the AI to do specific things; maybe build an big air attack group,
or defend the base. It would also be possible to make Al Ice come up with plans itself; how to
attack, what units to build, etc.

The other aspect of team play is the enemy; how many players are there in a team and are there many
teams? A good question would then be who to attack and could be implemented with the use of goals.

\subsubsection{Playing at the same level as the opponent}
Instead of always trying to do it's best the AI could be made more stupid when the opponent isn't as
good with the goal to create a 50-50\% success rate. Creating an AI like this would make it more fun
to play against, and as you get better it will also.